%!TEX root = tesis.tex
\section*{Introducción}
El campo del desarrollo de software está en constante evolución. 
Hoy en día, las herramientas que utilizan los desarrolladores 
son cada vez más poderosas, útiles e inteligentes 
y gracias a ello la calidad del software también ha aumentado.

Dentro de la Inteligencia Artificial (AI) se han explorado 
una variedad de enfoques para apoyar el desarrollo de software.
El razonamiento basado en casos (CBR), se puede utilizar para
mantener una biblioteca de rutinas de código (objetos, métodos),
seleccionar las rutinas de código 
que mejor se adapten a las especificaciones del usuario,
y presentar esas opciones al desarrollador de software.
Alternativamente, a través de factores genéticos y evolutivos
programación, el código puede ser mutado y probado para
Mejoras. Si se mejora, el nuevo código se convierte en una base
para la próxima generación de código. Los cambios aleatorios pueden
potencialmente conducir a un código que es más conciso, más
eficiente o más correcto. 

Los modelos de procesamiento del lenguaje natural (NLP) 
se han utilizado ampliamente para estudiar la relación entre palabras. 

Inspirados por modelos como n-gram, desarrollamos un modelo 
para analizar el código fuente a través de su árbol de sintaxis abstracta (AST). 

Uno de los objetivos dentro del campo de la inteligencia artificial 
es crear sistemas que tengan la capacidad de generar código fuente.

Este estudio tiene aplicaciones en generación de código fuente, 
finalización de código y análisis e investigación forense de software. 

El estudio también beneficia a los desarrolladores de software y programadores 
que se esfuerzan por mejorar la eficiencia del código 
y acelerar el proceso de desarrollo. 

Con el análisis de código fuente, 
el kit de herramientas de lenguaje natural (NLTK), 
que es muy útil en PNL, se ve severamente limitado 
por su incapacidad para manejar 
las propiedades semánticas y sintácticas de los códigos fuente. 

Por lo tanto, procesamos los conjuntos de datos del código fuente 
como árboles de sintaxis abstracta (AST) 
en lugar del texto del código fuente en sí 
para aprovechar la estructura rica en información del AST. 

El modelo propuesto se basa en las arquitecturas 
Long Short-Term Memory (LSTM) y Multiple Layer Perceptron (MLP) 
basadas en el aprendizaje profundo. 

Los resultados de nuestra evaluación 
intrínseca en un corpus de proyectos de Python 
han demostrado su capacidad para predecir de manera efectiva 
una secuencia de código fuente 
y muestran una mejora con respecto al trabajo anterior en este campo.




El uso de entornos de desarrollo integrados (IDE), 
se ha convertido en una práctica habitual ya que 
las funciones adicionales que aporta son muy útiles. 

La función de autocompletado 
se ha convertido en una característica muy útil en los IDE, 
ya que sugiere el próximo token probable basado en el código 
ya listo en el contexto de escritura. 

El trabajo anterior sobre la finalización del código 
se basó en la información del tipo de tiempo de compilación [1] 
para predecir el siguiente token. 

Esta estrategia es eficiente con lenguajes de tipado estático 
como C ++, java y otros lenguajes de tipo definido, 
pero encuentra dificultades cuando se usa con lenguajes de tipado dinámico 
como Python y JavaScript porque no hay declaración de tipo en estos últimos lenguajes. 




